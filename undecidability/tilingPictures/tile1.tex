
\documentclass[12pt]{article}

\pagestyle{empty}


\oddsidemargin 0pt
\evensidemargin 0 pt
\topmargin -1in
\headsep 20pt
\footskip 20pt
\textheight 8.5in
\textwidth 6.25in

\usepackage{tikz}
\usetikzlibrary{arrows,shapes,automata,backgrounds,petri}
\usepackage{amsmath,amsthm,stmaryrd,ifsym}

\usepackage{amssymb}
\usepackage{pifont}
%\input xy
%\xyoption{all}


\usepackage{pxfonts}
\usepackage{latexsym}


\newcommand{\rem}[1]{\relax}
% \beamertemplatetransparentcovereddynamic
               % overlays that are upcoming are transparent, in


%\xyoption{arc}

%\newcommand{\hearts}{\textcolor{red}{\varheartsuit}}
%\newcommand{\diamonds}{\textcolor{red}{\vardiamondsuit}}
%\renewcommand{\heart}{\hearts}
%\renewcommand{\diamonds}{\diamonds}
%\newcommand{\clubs}{\clubsuit}

%\newcommand{\xbar}{\overline{x}}
\newcommand{\spade}{\spadesuit}
\newcommand{\diamonds}{\textcolor{red}{\vardiamondsuit}}
\newcommand{\monus} {$--$\hspace{-.07in}\raisebox{1.0ex}
{$\scriptstyle{\bullet}$}\hspace{0in}} %.2
\newcommand{\divides}{\mbox{\em div}}
\newcommand{\hash}{\mbox{\tt \#}}
\newcommand{\one}{\mbox{\tt 1}}
\newcommand{\addone}{\lozenge}
\newcommand{\addhash}{\spade}
\newcommand{\passthrough}{\bigcirc}%{\mbox{\tt 0}}
%\newcommand{\spade}{\spadesuit}
%\newcommand{\club}{\clubsuit}
%\newcommand{\heart}{\heartsuit}
\newcommand{\leftendmarker}{\talloblong}%{\triangleright}%{\spade}
\newcommand{\emptyreg}{\Box}%{\diamondsuit}
\newcommand{\thislineends}{} %%{\textcolor{blue}{\star}}
\newcommand{\numberone}{\emph{1}}
\newcommand{\trileft}{\ }

\newcommand{\tvec}{\vec{t}}
\newcommand{\uvec}{\vec{u}}
\newcommand{\Tile}{\mbox{\textit{Tile}}}

\newcommand{\inverse}[1]{{#1}}



%\renewcommand{\dar}{\textcolor{blue}{\star}}


\newcommand{\tile}[4]
{
 \begin{tikzpicture}
\foreach \x in {0}
\foreach \y in {0}
{
\draw (\x, \y)    rectangle ++(2,2);
};
\draw  (1,1.7) node{\protect{$#1$}};  %% north
\draw (1.6,1) node{\protect{$#4$}}; %% east
\draw  (1,.3) node{\protect{$#2$}};  %% south
\draw (.4,1) node{\protect{$#3$}};%% west
\end{tikzpicture}
}

\newcommand{\tilegreen}[4]
 {
 \begin{tikzpicture}
  \filldraw[fill=green!20,draw=green!50!black] (0,0)    rectangle ++(2,2);
\foreach \x in {0}
\foreach \y in {0}
{
\draw (\x, \y)    rectangle ++(2,2);
};
\draw  (1,1.7) node{\protect{$#1$}};  %% north
\draw (1.7,1) node{\protect{$#4$}}; %% east
\draw  (1,.3) node{\protect{$#2$}};  %% south
\draw (.3,1) node{\protect{$#3$}};%% west
\end{tikzpicture}
}


\begin{document}

%%  pdftocairo  -x 50 -y 50  -H 140  -png tile1.pdf tile14
%% where tile1.pdf is the source file
%% tile14.png is the target
%% using the settings below and going for one line. -H 140
%% two lines -H 300
%% 
%% \oddsidemargin 0pt
%% \evensidemargin 0 pt
%% \topmargin -1in
%% \headsep 20pt
%% \footskip 20pt
%% \textheight 8.5in
%% \textwidth 6.25in




\begin{flushleft}
\tile{\one}{\one}{\diamonds}{\diamonds}\, 
\tile{\hash}{\hash}{\diamonds}{\diamonds}\,
\tile{\one}{\thislineends}{\diamonds}{\trileft}
\end{flushleft}

%\

\vfil

\eject

\begin{flushleft}
\tile{}{}{}{}
\tilegreen{\numberone}{start}{\leftendmarker}{\trileft}
%\tile{\thislineends}{}{\trileft}{}
%\tile{}{\thislineends}{\thislineends}{}
\end{flushleft}


\begin{flushleft}
\tile{\one}{\one}{\diamonds}{\diamonds}
\tile{\hash}{\hash}{\diamonds}{\diamonds}
\tile{\one}{\thislineends}{\diamonds}{\trileft}
\end{flushleft}


\begin{flushleft}
\tile{\one}{\one}{\spade}{\spade}
\tile{\hash}{\hash}{\spade}{\spade}
\tile{\hash}{\thislineends}{\spade}{\trileft}
%\tile{\thislineends}{}{\trileft}{}
\end{flushleft}

\begin{flushleft}
\tile{\one}{\one}{\inverse{\one}}{\inverse{\one}}
\tile{\one}{\hash}{\inverse{\hash}}{\inverse{\one}}
\tile{\hash}{\one}{\inverse{\one}}{\inverse{\hash}}
\tile{\hash}{\hash}{\inverse{\hash}}{\inverse{\hash}}
\tile{\trileft}{\one}{\inverse{\one}}{\inverse{\trileft}}
\tile{\trileft}{\hash}{\inverse{\hash}}{\inverse{\trileft}}
\end{flushleft}

\begin{flushleft}
%\tile{i+n}{i}{\leftendmarker}{\copyright}
\tile{\one}{\one}{\copyright}{\copyright}
\tile{\hash}{\hash}{\copyright}{\copyright}
\tile{\one}{\one}{\copyright}{}
\tile{\hash}{\hash}{\copyright}{}
%\tile{}{}{\copyright}{}
% check to be sure if we need the tile above
\end{flushleft}

\[
\begin{array}{c@{\qquad}c}
\mbox{When  instruction $i$ is  $\one\hash$} & \mbox{When  instruction $i$ is  $\one\hash\hash$}\\
\tile{i+1}{i}{\leftendmarker}{\diamonds} & \tile{i+1}{i}{\leftendmarker}{\spade}\\
\end{array}
\]


\[
\begin{array}{c@{\qquad}c}
\mbox{When  instruction $i$ is  $\one^m\hash\hash\hash$} & \mbox{When  instruction $i$ is  $\one^m\hash\hash\hash\hash$}\\
\tile{i+m}{i}{\leftendmarker}{\copyright} & \tile{i-m}{i}{\leftendmarker}{\copyright}\\
\end{array}
\]


\rem{
When  instruction $i$ is  $\one\hash$
\begin{flushleft}
\tile{i+1}{i}{\leftendmarker}{\diamonds}
\quad
\end{flushleft}

\vfil


When  instruction $i$ is  $\one\hash\hash$
\begin{flushleft}
\tile{i+1}{i}{\leftendmarker}{\spade}
\quad
\end{flushleft}


\vfil

\[
\begin{array}{c}
\mbox{When  instruction $i$ is  $\one^n\hash\hash\hash$ (and similarly if it is  $\one^n\hash\hash\hash\hash$)}
\begin{flushleft}
\tile{i+n}{i}{\leftendmarker}{\copyright}
\tile{\spade}{\spade}{\copyright}{\copyright}
\tile{\diamonds}{\diamonds}{\copyright}{\copyright}
\tile{\spade}{\spade}{\copyright}{}
\tile{\diamonds}{\diamonds}{\copyright}{}
%\tile{}{}{\copyright}{}
% check to be sure if we need the tile above
\end{flushleft}
}

\vfil

\[
\begin{array}{c}
 \mbox{When  instruction $i$ is  $\one\hash\hash\hash\hash\hash$}\\
%\tile{\Simley{-1}}{i}{\leftendmarker}{\thislineends}
\tile{i+1}{i}{\leftendmarker}{\thislineends} \qquad
\tile{i+2}{i}{\leftendmarker}{\inverse{\one}}\qquad
\tile{i+3}{i}{\leftendmarker}{\inverse{\hash}}
\end{array}
\]


%Since $p$ is tidy, all numbers on the tops of

% these tiles are between $1$ and $n+1$,

%where $n$ is the number of instructions in $p$



\end{document}

